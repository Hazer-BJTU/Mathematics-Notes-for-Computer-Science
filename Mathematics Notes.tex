\documentclass[UTF8]{ctexart}
\usepackage{geometry}
\usepackage{amsmath}
\usepackage{lmodern}
\usepackage{amsmath,amsthm,amssymb,amsfonts}

\geometry{a4paper,scale=0.75} %设置页边距

\title{\Huge{Mathematics Notes} \\ \huge{for} \\ \Huge{Computer Science} \\ \Huge{Information Technology}}
\author{\Large Hazer-BJTU}
\date{\Large 2024 / 2 / 16}

\begin{document}
\maketitle
\newpage
\tableofcontents %生成目录
\newpage

\CTEXsetup[format={\Large\bfseries}]{section} %设置段落标题左对齐

\section{深度学习中的线性代数/概率论}

\subsection{多元函数微分}
考虑定义在$\mathbb{R}^n$上的函数$f$,其输出为一个向量$\mathbf{y}\in \mathbb{R}^m$,如果存在线性函数$L$,使得:
\begin{large}
    \begin{equation}
        \begin{aligned}
            f(\mathbf{x}+\mathbf{h})=f(\mathbf{x})+L(\mathbf{h})+O\left (\left \| \mathbf{h} \right \|_2\right )
            \nonumber
        \end{aligned}
    \end{equation}
\end{large}
其中线性函数$L$满足:
\begin{large}
    \begin{equation}
        \begin{aligned}
            &L(\mathbf{x}+\mathbf{y})=L(\mathbf{x})+L(\mathbf{y}) \\
            &L(\lambda \cdot \mathbf x)=\lambda \cdot L(\mathbf x), \lambda \in \mathbb{R}
            \nonumber 
        \end{aligned}
    \end{equation}
\end{large}
那么我们就认为该函数$f$是\textbf{可微的},一般来说,我们可以将线性函数$L$简单理解为线性变换,如果我们限制函数$f$的输出为一个实数$y\in\mathbb{R}$,则微分也可以被表示为如下形式:
\begin{large}
    \begin{equation}
        \begin{aligned}
            f(\mathbf{x}+\mathbf{h})=f(\mathbf{x})+\mathbf{w}^\top\mathbf{h}+O\left (\left \| \mathbf{h} \right \|_2\right ), \mathbf{w}\in \mathbb{R}^n
            \nonumber
        \end{aligned}
    \end{equation}
\end{large}
一个基本的事实是可微$\Rightarrow$偏导数存在,因为:
\begin{large}
    \begin{equation}
        \begin{aligned}
            &\frac{f(\mathbf{x}+\Delta\mathbf{x}_{i})-f(\mathbf{x})}{\Delta\mathbf{x}_i} = \frac{\mathbf{w}_i \cdot \Delta\mathbf{x}_i}{\Delta\mathbf{x}_i}+\frac{O(\Delta\mathbf{x}_i)}{\Delta\mathbf{x}_i}= \mathbf{w}_i+\frac{O(\Delta\mathbf{x}_i)}{\Delta\mathbf{x}_i} \\
            &\Rightarrow \lim_{\Delta\mathbf{x}_{i}\to 0} \frac{f(\mathbf{x}+\Delta\mathbf{x}_{i})-f(\mathbf{x})}{\Delta\mathbf{x}_i}=\mathbf{w}_i+\lim_{\Delta\mathbf{x}_{i}\to 0}\frac{O(\Delta\mathbf{x}_i)}{\Delta\mathbf{x}_i}=\mathbf{w}_i \\
            &\Rightarrow \frac{\partial f}{\partial \mathbf{x}_i}=\mathbf{w}_i 
            \nonumber
        \end{aligned}
    \end{equation}
\end{large}
由此可见,实际上向量$\mathbf{w}$就是由函数$f$关于各分量的偏导数构成的:
\begin{large}
    \begin{equation}
        \begin{aligned}
            \mathbf{w}=\left ( \frac{\partial f}{\partial \mathbf{x}_0},\frac{\partial f}{\partial \mathbf{x}_1},\frac{\partial f}{\partial \mathbf{x}_2},\dots,\frac{\partial f}{\partial \mathbf{x}_n} \right )
            \nonumber
        \end{aligned}
    \end{equation}
\end{large}
定义对于向量$\mathbf{x}\in\mathbb{R}^n$: $\mathrm{d}\mathbf{x}=(\mathrm{d}\mathbf{x}_0,\mathrm{d}\mathbf{x}_1,\mathrm{d}\mathbf{x}_2,\dots,\mathrm{d}\mathbf{x}_n)$,则根据全微分公式可以得出如下关系:
\begin{large}
    \begin{equation}
        \begin{aligned}
            &\mathrm{d}\mathbf{x}^\top\mathbf{x}=2\mathbf{x}^\top\mathrm{d}\mathbf{x} \\
            &\mathrm{d}(\mathbf{x}+\mathbf{y})=\mathrm{d}\mathbf{x}+\mathrm{d}\mathbf{y} \\
            &\mathrm{d}{A\mathbf{x}}=A\mathrm{d}\mathbf{x} \\
            & 
            \nonumber
        \end{aligned}
    \end{equation}
\end{large}
留给读者自证。

\subsection{线性回归模型的解析解}
一般的线性模型可以被描述为以下形式,其中 $\hat{y}\in \mathbb{R}, \mathbf{x}\in \mathbb{R}^d, \mathbf{w}\in \mathbb{R}^d$ :
\begin{large}
    \begin{equation}
        \begin{aligned}
            \hat{y}= \mathbf{w}^\top \mathbf{x}+\mathbf{b}
            \nonumber
        \end{aligned}
    \end{equation}
\end{large}
而对于批量的样本数据,使用 $\mathbf{X}\in \mathbb{R}^{n\times d}$ 表示 $n$ 组样本,$\hat{\mathbf{Y}}\in \mathbb{R}^{n}$ 表示对于数据集上所有样本的预测结果向量,则可以进行如下矩阵表示:
\begin{large}
    \begin{equation}
        \begin{aligned}
            \hat{\mathbf{Y}}=\mathbf{X}\mathbf{w}+\mathbf{B}
            \nonumber
        \end{aligned}
    \end{equation}
\end{large}
对于真实的数据$Y$,线性回归要求我们最小化损失$\left \| \hat{\mathbf{Y}}-\mathbf{Y} \right \|_2$,这是一个十分简单的优化问题,存在解析解,证明如下:
\begin{large}
    \begin{equation}
        \begin{aligned}
            \left \| \hat{\mathbf{Y}}-\mathbf{Y} \right \|_2 &= \sqrt{(\hat{\mathbf{Y}}-\mathbf{Y})^\top(\hat{\mathbf{Y}}-\mathbf{Y})} \\
            &= \sqrt{(\mathbf{X}\mathbf{w}+\mathbf{B}-\mathbf{Y})^\top(\mathbf{X}\mathbf{w}+\mathbf{B}-\mathbf{Y})}
            \nonumber
        \end{aligned}
    \end{equation}
\end{large}
故问题转化为最小化$(\mathbf{X}\mathbf{w}+\mathbf{B}-\mathbf{Y})^\top(\mathbf{X}\mathbf{w}+\mathbf{B}-\mathbf{Y})$,这是一个二次型,我们对于$\mathbf{w}$求导:
\begin{large}
    \begin{equation}
        \begin{aligned}
            \mathrm{d}(\mathbf{X}\mathbf{w}+\mathbf{B}-\mathbf{Y})^\top(\mathbf{X}\mathbf{w}+\mathbf{B}-\mathbf{Y}) &= 2(\mathbf{X}\mathbf{w}+\mathbf{B}-\mathbf{Y})^\top\mathrm{d}(\mathbf{X}\mathbf{w}+\mathbf{B}-\mathbf{Y}) \\
            &= 2(\mathbf{X}\mathbf{w}+\mathbf{B}-\mathbf{Y})^\top\mathbf{X}\mathrm{d}\mathbf{w} \\
            &= 0
            \nonumber
        \end{aligned}
    \end{equation}
\end{large}
故可以得到:
\begin{large}
    \begin{equation}
        \begin{aligned}
            (\mathbf{X}\mathbf{w}+\mathbf{B}-\mathbf{Y})^\top\mathbf{X}=\mathbf{O}
            \nonumber
        \end{aligned}
    \end{equation}
\end{large}
等式两边同时取转置可知:
\begin{large}
    \begin{equation}
        \begin{aligned}
            &\mathbf{X}^\top(\mathbf{X}\mathbf{w}+\mathbf{B}-\mathbf{Y})=\mathbf{O} \\
            &\mathbf{X}^\top\mathbf{X}\mathbf{w}=\mathbf{X}^\top(\mathbf{Y}-\mathbf{B}) \\
            &\mathbf{w}=(\mathbf{X}^\top\mathbf{X})^{-1}\mathbf{X}^\top(\mathbf{Y}-\mathbf{B})
            \nonumber
        \end{aligned}
    \end{equation}
\end{large}
即可得到参数的最优解,前提是矩阵$\mathbf{X}^\top\mathbf{X}$可逆。

\end{document}