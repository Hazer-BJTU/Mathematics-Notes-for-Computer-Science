\documentclass[UTF8]{book}
\usepackage{ctex}
\usepackage{geometry}
\usepackage{amsmath}
\usepackage{lmodern}
\usepackage{amsmath,amsthm,amssymb,amsfonts}
\usepackage{tikz}
\usepackage{algorithm}
\usepackage{algorithmic}
\usetikzlibrary{cd}

\geometry{a4paper,scale=0.75} %设置页边距

\title{\Huge{Mathematics Notes} \\ \huge{for} \\ \Huge{Computer Science} \\ \Huge{Information Technology}}
\author{\Large Hazer-BJTU}
\date{\Large 2024 / 2 / 16}

\begin{document}
\maketitle
\tableofcontents %生成目录
\newpage

% \CTEXsetup[format={\Large\bfseries}]{section} %设置段落标题左对齐

\section{深度学习中的线性代数/概率论}

\subsection{多元函数微分}
考虑定义在$\mathbb{R}^n$上的函数$f$,其输出为一个向量$\mathbf{y}\in \mathbb{R}^m$,如果存在线性函数$L$,使得:
\begin{large}
    \begin{equation}
        \begin{aligned}
            f(\mathbf{x}+\mathbf{h})=f(\mathbf{x})+L(\mathbf{h})+O\left (\left \| \mathbf{h} \right \|_2\right )
            \nonumber
        \end{aligned}
    \end{equation}
\end{large}
其中线性函数$L$满足:
\begin{large}
    \begin{equation}
        \begin{aligned}
            &L(\mathbf{x}+\mathbf{y})=L(\mathbf{x})+L(\mathbf{y}) \\
            &L(\lambda \cdot \mathbf x)=\lambda \cdot L(\mathbf x), \lambda \in \mathbb{R}
            \nonumber 
        \end{aligned}
    \end{equation}
\end{large}
那么我们就认为该函数$f$是\textbf{可微的},一般来说,我们可以将线性函数$L$简单理解为线性变换,如果我们限制函数$f$的输出为一个实数$y\in\mathbb{R}$,则微分也可以被表示为如下形式:
\begin{large}
    \begin{equation}
        \begin{aligned}
            f(\mathbf{x}+\mathbf{h})=f(\mathbf{x})+\mathbf{w}^\top\mathbf{h}+O\left (\left \| \mathbf{h} \right \|_2\right ), \mathbf{w}\in \mathbb{R}^n
            \nonumber
        \end{aligned}
    \end{equation}
\end{large}
一个基本的事实是可微$\Rightarrow$偏导数存在,因为:
\begin{large}
    \begin{equation}
        \begin{aligned}
            &\frac{f(\mathbf{x}+\mathbf{h}_i)-f(\mathbf{x})}{\Delta\mathbf{x}_i} = \frac{\mathbf{w}_i \cdot \Delta\mathbf{x}_i}{\Delta\mathbf{x}_i}+\frac{O(\Delta\mathbf{x}_i)}{\Delta\mathbf{x}_i}= \mathbf{w}_i+\frac{O(\Delta\mathbf{x}_i)}{\Delta\mathbf{x}_i} \\
            &\Rightarrow \lim_{\Delta\mathbf{x}_{i}\to 0} \frac{f(\mathbf{x}+\mathbf{h}_{i})-f(\mathbf{x})}{\Delta\mathbf{x}_i}=\mathbf{w}_i+\lim_{\Delta\mathbf{x}_{i}\to 0}\frac{O(\Delta\mathbf{x}_i)}{\Delta\mathbf{x}_i}=\mathbf{w}_i \\
            &\Rightarrow \frac{\partial f}{\partial \mathbf{x}_i}=\mathbf{w}_i 
            \nonumber
        \end{aligned}
    \end{equation}
\end{large}
由此可见,实际上向量$\mathbf{w}$就是由函数$f$关于各分量的偏导数构成的:
\begin{large}
    \begin{equation}
        \begin{aligned}
            \mathbf{w}=\left ( \frac{\partial f}{\partial \mathbf{x}_1},\frac{\partial f}{\partial \mathbf{x}_2},\frac{\partial f}{\partial \mathbf{x}_3},\dots,\frac{\partial f}{\partial \mathbf{x}_n} \right )^\top
            \nonumber
        \end{aligned}
    \end{equation}
\end{large}
定义对于向量$\mathbf{x}\in\mathbb{R}^n$: $\mathrm{d}\mathbf{x}=(\mathrm{d}\mathbf{x}_1,\mathrm{d}\mathbf{x}_2,\mathrm{d}\mathbf{x}_3,\dots,\mathrm{d}\mathbf{x}_n)$,则根据全微分公式可以得出如下关系:
\begin{large}
    \begin{equation}
        \begin{aligned}
            &\mathrm{d}\mathbf{x}^\top\mathbf{x}=2\mathbf{x}^\top\mathrm{d}\mathbf{x} \\
            &\mathrm{d}(\mathbf{x}+\mathbf{y})=\mathrm{d}\mathbf{x}+\mathrm{d}\mathbf{y} \\
            &\mathrm{d}{\mathbf{A}\mathbf{x}}=\mathbf{A}\mathrm{d}\mathbf{x} \\
            &\mathrm{d}{\mathbf{x}^\top \mathbf{A} \mathbf{x}}=2\mathbf{x}^\top \mathbf{A} \mathrm{d}\mathbf{x}
            \nonumber
        \end{aligned}
    \end{equation}
\end{large}
在此只证明最后一条,注意到:
\begin{large}
    \begin{equation}
        \begin{aligned}
            &\mathbf{x}^\top \mathbf{A} \mathbf{x}=\sum_{i=1}^{n}\sum_{j=1}^{n}\mathbf{A}_{i,j}\mathbf{x}_i\mathbf{x}_j \\
            \frac{\partial}{\partial \mathbf{x}_i}&\mathbf{x}^\top \mathbf{A} \mathbf{x}=2\mathbf{A}_{i,i}\mathbf{x}_i+2\sum_{1 \le j \le n, j\not = i}{\mathbf{A}_{i,j}\mathbf{x}_j}=2\sum_{j=1}^{n}\mathbf{A}_{i,j}\mathbf{x}_j \\
            \Rightarrow \mathrm{d}&\mathbf{x}^\top \mathbf{A} \mathbf{x}=2\sum_{i=1}^{n}{\sum_{j=1}^{n}\mathbf{A}_{i,j}\mathbf{x}_j\mathrm{d}\mathbf{x}_i}=2\mathbf{x}^\top \mathbf{A}\mathrm{d}\mathbf{x}
            \nonumber
        \end{aligned}
    \end{equation}
\end{large}
与一元函数同理,如果上述函数$f$满足二阶偏导数连续的条件,则我们也可以利用Hessian矩阵做出更高阶的估计:
\begin{large}
    \begin{equation}
        \begin{aligned}
            f(\mathbf{x}+\mathbf{h})=f(\mathbf{x})+\mathbf{w}^\top \mathbf{h}+\frac{1}{2}\mathbf{h}^\top \mathbf{H}\mathbf{h}+O(\left \| \mathbf{h} \right \|_2^2)
            \nonumber
        \end{aligned}
    \end{equation}
\end{large}
其中Hessian矩阵的形式为:
\begin{large}
    \begin{equation}
        \begin{aligned}
            \mathbf{H}_{i,j}=\frac{\partial^2f}{\partial\mathbf{x}_i\partial\mathbf{x}_j}
            \nonumber
        \end{aligned}
    \end{equation}
\end{large}
        

\subsection{线性回归模型的解析解}
一般的线性模型可以被描述为以下形式,其中 $\hat{y}\in \mathbb{R}, \mathbf{x}\in \mathbb{R}^d, \mathbf{w}\in \mathbb{R}^d$ :
\begin{large}
    \begin{equation}
        \begin{aligned}
            \hat{y}= \mathbf{w}^\top \mathbf{x}+\mathbf{b}
            \nonumber
        \end{aligned}
    \end{equation}
\end{large}
而对于批量的样本数据,使用 $\mathbf{X}\in \mathbb{R}^{n\times d}$ 表示 $n$ 组样本,$\hat{\mathbf{Y}}\in \mathbb{R}^{n}$ 表示对于数据集上所有样本的预测结果向量,则可以进行如下矩阵表示:
\begin{large}
    \begin{equation}
        \begin{aligned}
            \hat{\mathbf{Y}}=\mathbf{X}\mathbf{w}+\mathbf{B}
            \nonumber
        \end{aligned}
    \end{equation}
\end{large}
对于真实的数据$Y$,线性回归要求我们最小化损失$\left \| \hat{\mathbf{Y}}-\mathbf{Y} \right \|_2^2$,这是一个十分简单的优化问题,存在解析解,证明如下:
\begin{large}
    \begin{equation}
        \begin{aligned}
            \left \| \hat{\mathbf{Y}}-\mathbf{Y} \right \|_2^2 &= (\hat{\mathbf{Y}}-\mathbf{Y})^\top(\hat{\mathbf{Y}}-\mathbf{Y}) \\
            &= (\mathbf{X}\mathbf{w}+\mathbf{B}-\mathbf{Y})^\top(\mathbf{X}\mathbf{w}+\mathbf{B}-\mathbf{Y})
            \nonumber
        \end{aligned}
    \end{equation}
\end{large}
故问题转化为最小化$(\mathbf{X}\mathbf{w}+\mathbf{B}-\mathbf{Y})^\top(\mathbf{X}\mathbf{w}+\mathbf{B}-\mathbf{Y})$,这是一个二次型,我们对于$\mathbf{w}$求导:
\begin{large}
    \begin{equation}
        \begin{aligned}
            \mathrm{d}(\mathbf{X}\mathbf{w}+\mathbf{B}-\mathbf{Y})^\top(\mathbf{X}\mathbf{w}+\mathbf{B}-\mathbf{Y}) &= 2(\mathbf{X}\mathbf{w}+\mathbf{B}-\mathbf{Y})^\top\mathrm{d}(\mathbf{X}\mathbf{w}+\mathbf{B}-\mathbf{Y}) \\
            &= 2(\mathbf{X}\mathbf{w}+\mathbf{B}-\mathbf{Y})^\top\mathbf{X}\mathrm{d}\mathbf{w} \\
            &= 0
            \nonumber
        \end{aligned}
    \end{equation}
\end{large}
故可以得到:
\begin{large}
    \begin{equation}
        \begin{aligned}
            (\mathbf{X}\mathbf{w}+\mathbf{B}-\mathbf{Y})^\top\mathbf{X}=\mathbf{O}
            \nonumber
        \end{aligned}
    \end{equation}
\end{large}
等式两边同时取转置可知:
\begin{large}
    \begin{equation}
        \begin{aligned}
            &\mathbf{X}^\top(\mathbf{X}\mathbf{w}+\mathbf{B}-\mathbf{Y})=\mathbf{O} \\
            &\mathbf{X}^\top\mathbf{X}\mathbf{w}=\mathbf{X}^\top(\mathbf{Y}-\mathbf{B}) \\
            &\mathbf{w}=(\mathbf{X}^\top\mathbf{X})^{-1}\mathbf{X}^\top(\mathbf{Y}-\mathbf{B})
            \nonumber
        \end{aligned}
    \end{equation}
\end{large}
即可得到参数的最优解,前提是矩阵$\mathbf{X}^\top\mathbf{X}$可逆。

\subsection{SVD奇异值分解}
一般来说,任何实矩阵$\mathbf{A}\in \mathbb{R}^{n\times m}$都可以被无条件地分解为如下三个矩阵的乘积:
\begin{large}
    \begin{equation}
        \begin{aligned}
            \mathbf{A}_{n\times m}=\mathbf{U}_{n\times n}\mathbf{\Sigma}_{n\times m}\mathbf{V}_{m\times m}^\top
            \nonumber
        \end{aligned}
    \end{equation}
\end{large}
其中$\mathbf{U},\mathbf{V}$均为正交矩阵,并且$\mathbf{\Sigma}$满足:
\begin{large}
    \begin{equation}
        \begin{aligned}
            \mathbf{\Sigma}_{i,j}=\begin{cases}
                \sqrt{\lambda_i} & i=j\\
                0 & i\not = j
               \end{cases} 
            \nonumber
        \end{aligned}
    \end{equation}
\end{large}
考虑$\mathbf{A}^\top \mathbf{A}$,这是一个实对称矩阵,故其一定可以被正交对角化,也即存在正交矩阵$\mathbf{V}$,使得:
\begin{large}
    \begin{equation}
        \begin{aligned}
            \mathbf{A}^\top \mathbf{A}=\mathbf{V}\mathbf{\Lambda}\mathbf{V}^\top
            \nonumber
        \end{aligned}
    \end{equation}
\end{large}
其中:
\begin{large}
    \begin{equation}
        \begin{aligned}
            \mathbf{\Lambda}_{m\times m}=\begin{bmatrix}
                \lambda_1 &  &  & \\
                 & \lambda_2 &  & \\
                 &  & \ddots  & \\
                 &  &  &\lambda_m
               \end{bmatrix}
            \nonumber
        \end{aligned}
    \end{equation}
\end{large}
考虑如下一组向量,我们断言它们之间是互相正交的:
\begin{large}
    \begin{equation}
        \begin{aligned}
            \frac{\mathbf{A}\mathbf{v}_1}{\sqrt{\lambda_1}},\frac{\mathbf{A}\mathbf{v}_2}{\sqrt{\lambda_2}},\frac{\mathbf{A}\mathbf{v}_3}{\sqrt{\lambda_3}},\dots,\frac{\mathbf{A}\mathbf{v}_m}{\sqrt{\lambda_m}}
            \nonumber
        \end{aligned}
    \end{equation}
\end{large}
证明如下:
\begin{large}
    \begin{equation}
        \begin{aligned}
            \frac{\mathbf{A}\mathbf{v}_i}{\sqrt{\lambda}_i} \cdot \frac{\mathbf{A}\mathbf{v}_j}{\sqrt{\lambda}_j} = \frac{\mathbf{v}_i^\top\mathbf{A}^\top\mathbf{A}\mathbf{v}_j}{\sqrt{\lambda_i\lambda_j}} 
            = \frac{\lambda_j\mathbf{v}_i^\top\mathbf{v}_j}{\sqrt{\lambda_i\lambda_j}} 
            = \begin{cases}
                1 & i=j \\
                0 & i\not = j
                \end{cases}
            \nonumber
        \end{aligned}
    \end{equation}
\end{large}
若$m\ge n$,考虑如下矩阵:
\begin{large}
    \begin{equation}
        \begin{aligned}
            \mathbf{U}_{n\times n}=\left (  \frac{\mathbf{A}\mathbf{v}_1}{\sqrt{\lambda_1}},\frac{\mathbf{A}\mathbf{v}_2}{\sqrt{\lambda_2}},\frac{\mathbf{A}\mathbf{v}_3}{\sqrt{\lambda_3}},\dots,\frac{\mathbf{A}\mathbf{v}_n}{\sqrt{\lambda_n}}\right )
            \nonumber
        \end{aligned}
    \end{equation}
\end{large}
根据上述证明,$\mathbf{U}$是正交矩阵,并且满足:
\begin{large}
    \begin{equation}
        \begin{aligned}
            &\mathbf{U}\mathbf{\Sigma}=\mathbf{A}\mathbf{V} \\
            &\mathbf{A}=\mathbf{A}\mathbf{V}\mathbf{V}^\top=\mathbf{U}\mathbf{\Sigma}\mathbf{V}^\top
            \nonumber
        \end{aligned}
    \end{equation}
\end{large}
若$m<n$,我们可以反过来对$\mathbf{A}^\top$做奇异值分解,也可以得到相同的结果,奇异值分解告诉我们:任何线性变换都可以被分解为一次旋转(旋转、反射或其复合),一次维度变换及拉伸,一次旋转的复合。除此之外,其还可以被用于求一般矩阵的“逆”:
\begin{large}
    \begin{equation}
        \begin{aligned}
            &\mathbf{A}=\mathbf{U}\mathbf{\Sigma}\mathbf{V}^\top \\
            &\mathbf{A}^+=\mathbf{V}\mathbf{\Sigma}^+\mathbf{U}^\top
            \nonumber
        \end{aligned}
    \end{equation}
\end{large}
其中$\mathbf{\Sigma}^+$由将$\mathbf{\Sigma}$中非零元素取倒数后再转置得到。

\subsection{极大似然估计与最小化交叉熵损失}

\section{算法/基础数学}

\subsection{离散傅里叶变换DFT与快速傅里叶变换FFT}
对于数列$\left \{ a_n \right \}, \left \{ b_n \right \},0 \le n < N$,我们可以如下定义其离散卷积:
\begin{large}
    \begin{equation}
        \begin{aligned}
            &\left ( a*b \right )_k=\sum_{i=0}^{k}{a_ib_{k-i}} & 0 \le k < N
            \nonumber
        \end{aligned}
    \end{equation}
\end{large}
我们记单位根\large{$e^{\frac{2k\pi i}{n}}=\omega_{n}^k$},则可以如下定义其离散傅里叶变换及其逆变换:
\begin{large}
    \begin{equation}
        \begin{aligned}
            &DFT(a)_k=\sum_{t=0}^{N-1}{a_t \cdot \omega_{N}^{-kt}} \\
            &a_k=\frac{1}{N}\sum_{t=0}^{N-1}{DFT(a)_t \cdot \omega_{N}^{kt}}
            \nonumber
        \end{aligned}
    \end{equation}
\end{large}
其中逆变换的证明如下:
\begin{large}
    \begin{equation}
        \begin{aligned}
            \frac{1}{N}\sum_{t=0}^{N-1}{DFT(a)_t \cdot \omega_{N}^{kt}} &= \frac{1}{N}\sum_{t=0}^{N-1}\left ( \sum_{u=0}^{N-1}a_u \cdot \omega_N^{-tu} \right ) \cdot \omega_{N}^{kt} \\
            &= \frac{1}{N}\sum_{t=0}^{N-1}\sum_{u=0}^{N-1}a_u \cdot \omega_N^{t(k-u)} \\
            &= \frac{1}{N}\sum_{u=0}^{N-1}\sum_{t=0}^{N-1}a_u \cdot \omega_N^{t(k-u)}
            \nonumber
        \end{aligned}
    \end{equation}
\end{large}
首先考虑如果$u=k$,则有下式成立:
\begin{large}
    \begin{equation}
        \begin{aligned}
            \omega_N^{t(k-u)}&=\omega_N^{0}=1 \\
            \sum_{t=0}^{N-1}a_u \cdot \omega_N^{t(k-u)}&=Na_u=Na_k
            \nonumber
        \end{aligned}
    \end{equation}
\end{large}
然后考虑如果$u \not = k$,注意到$\omega_N^N=1$,则有下式成立:
\begin{large}
    \begin{equation}
        \begin{aligned}
            \sum_{t=0}^{N-1}a_u \cdot \omega_N^{t(k-u)} &= a_u\sum_{t=0}^{N-1}\omega_N^{t(k-u)} \\
            &= a_u \cdot \frac{1-\omega_N^{N(k-u)}}{1-\omega_N^{k-u}} \\
            &= 0
            \nonumber
        \end{aligned}
    \end{equation}
\end{large}
故综上所述,逆变换得证:
\begin{large}
    \begin{equation}
        \begin{aligned}
            \frac{1}{N}\sum_{t=0}^{N-1}{DFT(a)_t \cdot \omega_{N}^{kt}}=\frac{1}{N} \cdot Na_k=a_k
            \nonumber
        \end{aligned}
    \end{equation}
\end{large}
接着我们引入卷积定理的离散形式:
\begin{large}
    \begin{equation}
        \begin{aligned}
            a*b=DFT^{-1}\left ( DFT(a) \odot DFT(b) \right )
            \nonumber
        \end{aligned}
    \end{equation}
\end{large}
为了证明上式,我们只需要证明:
\begin{large}
    \begin{equation}
        \begin{aligned}
            &(a*b)_k=DFT^{-1}\left ( DFT(a) \odot DFT(b) \right )_k & 0 \le k < N
            \nonumber
        \end{aligned}
    \end{equation}
\end{large}
利用定义展开右式,同理可证:
\begin{large}
    \begin{equation}
        \begin{aligned}
            &DFT^{-1}\left ( DFT(a) \odot DFT(b) \right )_k \\
            &= \frac{1}{N}\sum_{t=0}^{N-1}\left ( DFT(a) \odot DFT(b) \right )_t \cdot \omega_N^{kt}\\
            &= \frac{1}{N}\sum_{t=0}^{N-1} DFT(a)_t \cdot DFT(b)_t \cdot \omega_N^{kt} \\
            &= \frac{1}{N}\sum_{t=0}^{N-1} \left ( \sum_{n=0}^{N-1}{a_n \cdot \omega_{N}^{-tn}} \right ) \cdot \left ( \sum_{m=0}^{N-1}{b_m \cdot \omega_{N}^{-tm}} \right ) \cdot \omega_N^{kt} \\
            &= \frac{1}{N}\sum_{t=0}^{N-1} \left ( \sum_{n=0}^{N-1}\sum_{m=0}^{N-1}a_nb_m \cdot \omega_N^{-t(n+m)} \right ) \cdot \omega_N^{kt} \\
            &= \frac{1}{N}\sum_{n=0}^{N-1}\sum_{m=0}^{N-1}\sum_{t=0}^{N-1}a_nb_m \cdot \omega_N^{t(k-n-m)} \\
            &= \frac{1}{N}\sum_{n=0}^{N-1}\sum_{m=0}^{N-1}\left [ n+m=k \right ]Na_nb_m \\
            &= \sum_{n=0}^{k}a_nb_{k-n} \\
            &= (a*b)_k
            \nonumber
        \end{aligned}
    \end{equation}
\end{large}
快速傅里叶变换算法可以帮助我们高效地计算离散傅里叶变换:
\begin{large}
    \begin{equation}
        \begin{aligned}
            \begin{tikzcd}
                \left \{ a_n \right \} \arrow[r,"FFT"] & \left \{ DFT(a)_n \right \}
            \end{tikzcd}
            \nonumber
        \end{aligned}
    \end{equation}
\end{large}
结合卷积定理,我们得以加速多项式乘法至$O(n\log n)$时间复杂度:
\begin{large}
    \begin{equation}
        \begin{aligned}
            \begin{tikzcd}
                \left \{ a \right \}, \left \{ b \right \} \arrow[dd,"O(n\log n)"] \arrow[rr,"O(n^2)"] & & \left \{ a*b \right \} \\
                & & \\
                \left \{ DFT(a) \right \}, \left \{ DFT(b) \right \} \arrow[rr,"O(n)"] & & \left \{ DFT(a) \odot DFT(b) \right \} \arrow[uu,"O(n\log n)"]
            \end{tikzcd}
            \nonumber
        \end{aligned}
    \end{equation}
\end{large}
为了简化问题,此处我们只讨论$N=2^K, K\in \mathbb{N}$的简单情形,为了计算离散傅里叶变换,我们的目标是计算下列数值:
\begin{large}
    \begin{equation}
        \begin{aligned}
            &\mathbf{F}=\left [ F(w_N^{0}), F(w_N^{-1}), F(w_N^{-2}), \dots, F(w_N^{-(N-1)}) \right ] \\
            &F(x)=\sum_{t=0}^{N-1}a_tx^t \Rightarrow DFT(a)_k=F(\omega_N^{-k})
            \nonumber
        \end{aligned}
    \end{equation}
\end{large}
我们将函数$F(x)$拆分为如下两个部分,注意到当$N\not = 1$时其为偶数:
\begin{large}
    \begin{equation}
        \begin{aligned}
            F(x) &= a_0+a_1x+a_2x^2+\dots +a_{N-1}x^{N-1} \\
            &= (a_0+a_2x^2+\dots +a_{N-2}x^{N-2})+(a_1x+a_3x^3+\dots +a_{N-1}x^{N-1}) \\
            &= (a_0+a_2x^2+\dots +a_{N-2}x^{N-2})+x(a_1+a_3x^2+\dots +a_{N-1}x^{N-2}) \\
            &= A_{e}(x^2)+xA_{o}(x^2)
            \nonumber
        \end{aligned}
    \end{equation}
\end{large}
故问题转化为计算如下数值:
\begin{large}
    \begin{equation}
        \begin{aligned}
            &A_{e}(\omega_N^{-2k}), A_{o}(\omega_N^{-2k}) & 0 \le k <N
            \nonumber
        \end{aligned}
    \end{equation}
\end{large}
注意到:
\begin{large}
    \begin{equation}
        \begin{aligned}
            &\omega_N^{-2k}=\omega_{\frac{N}{2}}^{-k} \\
            &\omega_{\frac{N}{2}}^{-(k+\frac{N}{2})}=\omega_{\frac{N}{2}}^{-k} \\
            &\omega_{N}^{-(k+\frac{N}{2})}=-\omega_{N}^{-k}
            \nonumber
        \end{aligned}
    \end{equation}
\end{large}
所以我们实际上只需要计算如下数值:
\begin{large}
    \begin{equation}
        \begin{aligned}
            &\mathbf{A}_e=\left [ A_{e}(\omega_{\frac{N}{2}}^{0}),A_{e}(\omega_{\frac{N}{2}}^{-1}),A_{e}(\omega_{\frac{N}{2}}^{-2}),\dots ,A_{e}(\omega_{\frac{N}{2}}^{-(\frac{N}{2}-1)}) \right ] \\
            &\mathbf{A}_o=\left [ A_{o}(\omega_{\frac{N}{2}}^{0}),A_{o}(\omega_{\frac{N}{2}}^{-1}),A_{o}(\omega_{\frac{N}{2}}^{-2}),\dots ,A_{o}(\omega_{\frac{N}{2}}^{-(\frac{N}{2}-1)}) \right ]
            \nonumber
        \end{aligned}
    \end{equation}
\end{large}
便可以计算出所需要的数值:
\begin{large}
    \begin{equation}
        \begin{aligned}
            &\mathbf{F}\left [ k \right ]=\mathbf{A}_e\left [ k \right ]+\omega_N^{-k}\mathbf{A}_o\left [ k \right ] \\
            &\mathbf{F}\left [ k+\frac{N}{2} \right ]=\mathbf{A}_e\left [ k \right ]-\omega_N^{-k}\mathbf{A}_o\left [ k \right ] \\
            &0 \le k < \frac{N}{2}
            \nonumber
        \end{aligned}
    \end{equation}
\end{large}
在此我们只证明第二个算式:
\begin{large}
    \begin{equation}
        \begin{aligned}
            \mathbf{F}\left [ k+\frac{N}{2} \right ] &= F(\omega_N^{-(k+\frac{N}{2})}) \\
            &= A_e(\omega_N^{-2(k+\frac{N}{2})})+\omega_N^{-(k+\frac{N}{2})}A_o(\omega_N^{-2(k+\frac{N}{2})}) \\
            &= A_e(\omega_{\frac{N}{2}}^{-(k+\frac{N}{2})})-\omega_N^{-k}A_o(\omega_{\frac{N}{2}}^{-(k+\frac{N}{2})}) \\
            &= A_e(\omega_{\frac{N}{2}}^{-k})-\omega_N^{-k}A_o(\omega_{\frac{N}{2}}^{-k}) \\
            &= \mathbf{A}_e\left [ k \right ]-\omega_N^{-k}\mathbf{A}_o\left [ k \right ]
            \nonumber
        \end{aligned}
    \end{equation}
\end{large}
而$\mathbf{A}_e, \mathbf{A}_o$的计算又可以递归地使用上述方法,并且问题的规模在指数级地缩减,故我们可以利用FFT算法高效地实现离散傅里叶变换地计算。以下为对其算法时间复杂度的分析,假设问题规模为$N$时所对应的时间复杂度为$T(N)$,则根据上述讨论可知:
\begin{large}
    \begin{equation}
        \begin{aligned}
            T(N)=2 \cdot T(N/2)+N
            \nonumber
        \end{aligned}
    \end{equation}
\end{large}
我们不难归纳证明出:
\begin{large}
    \begin{equation}
        \begin{aligned}
            &T(N)=2^k \cdot T(N/2^k)+kN & k \in \mathbb{N}
            \nonumber
        \end{aligned}
    \end{equation}
\end{large}
因为:
\begin{large}
    \begin{equation}
        \begin{aligned}
            T(N)&=2^k \cdot T(N/2^k)+kN \\
            &=2^k \cdot \left [ 2\cdot T(N/2^{k+1})+N/2^k \right ]+kN \\
            &=2^{k+1} \cdot T(N/2^{k+1}) + (k+1)N
            \nonumber
        \end{aligned}
    \end{equation}
\end{large}
令$k=\log_2 N$,可知:
\begin{large}
    \begin{equation}
        \begin{aligned}
            T(N)=N \cdot T(1)+N\log_2 N=O\left ( N\log N \right )
            \nonumber
        \end{aligned}
    \end{equation}
\end{large}

\subsection{傅里叶变换}
如果一个具有周期$T$的函数$f(t)$在区间$\left [ 0,T \right ]$上满足Dirichlet条件,那么我们可以将其展开为如下形式的傅里叶级数:
\begin{large}
    \begin{equation}
        \begin{aligned}
            f(t) \sim \frac{a_0}{2}+\sum_{k=0}^{\infty}{\left ( a_k\cos \left (\frac{2\pi}{T}kt\right )+b_k\sin\left (\frac{2\pi}{T}kt\right ) \right )}
            \nonumber
        \end{aligned}
    \end{equation}
\end{large}
我们可以通过如下变换将其转化为更为简洁的形式,令$\frac{2\pi}{T}=\omega$,于是有:
\begin{large}
    \begin{equation}
        \begin{aligned}
            &\cos\left (\frac{2\pi}{T}kt\right )=\cos(\omega kt)=\frac{e^{i\omega kt}+e^{-i\omega kt}}{2} \\
            &\sin\left (\frac{2\pi}{T}kt\right )=\cos(\omega kt)=\frac{e^{i\omega kt}-e^{-i\omega kt}}{2i} \\
            \nonumber
        \end{aligned}
    \end{equation}
\end{large}
故而:
\begin{large}
    \begin{equation}
        \begin{aligned}
            &\frac{a_0}{2}+\sum_{k=0}^{\infty}{\left ( a_k\cos \left (\frac{2\pi}{T}kt\right )+b_k\sin\left (\frac{2\pi}{T}kt\right ) \right )} \\
            &=\frac{a_0}{2}+\sum_{k=0}^{\infty}\left (a_k \cdot \frac{e^{i\omega kt}+e^{-i\omega kt}}{2}+b_k \cdot \frac{e^{i\omega kt}-e^{-i\omega kt}}{2i} \right ) \\
            &=\frac{a_0}{2}+\sum_{k=0}^{\infty}\left ( \frac{a_k+ib_k}{2}e^{-i\omega kt}+\frac{a_k-ib_k}{2}e^{i\omega kt} \right ) \\
            &=\sum_{k=-\infty}^{+\infty} c_k \cdot e^{i\omega kt}
            \nonumber
        \end{aligned}
    \end{equation}
\end{large}
显然,其中:
\begin{large}
    \begin{equation}
        \begin{aligned}
            c_k=\left\{\begin{matrix}
                \displaystyle{\frac{a_k-ib_k}{2}} & k>0 \\
                & \\
                \displaystyle{\frac{a_0}{2}} & k=0 \\
                & \\
                \displaystyle{\frac{a_k+ib_k}{2}} & k<0
                \end{matrix}\right.
            \nonumber
        \end{aligned}
    \end{equation}
\end{large}
如果上述级数在区间$\left [ 0,T \right ]$内一致收敛于$f(t)$,这通常要求其极限函数的导数是平方可积的,那么我们可以通过如下方法求得其系数:
\begin{large}
    \begin{equation}
        \begin{aligned}
            c_k=\frac{1}{T}\int_{0}^{T}{f(t)e^{-i\omega kt}dt}
            \nonumber
        \end{aligned}
    \end{equation}
\end{large}
证明如下,由一致收敛性我们可以交换积分以及级数求和的次序:
\begin{large}
    \begin{equation}
        \begin{aligned}
            \int_{0}^{T}{f(t)e^{-i\omega kt}dt} &= \int_{0}^{T}{\sum_{n=-\infty}^{+\infty} c_n \cdot e^{i\omega nt} \cdot e^{-i\omega kt}dt} \\
            &= \sum_{n=-\infty}^{+\infty}{\int_{0}^{T} c_n \cdot e^{i\omega nt} \cdot e^{-i\omega kt}dt} \\
            &= \sum_{n=-\infty}^{+\infty}{c_n\int_{0}^{T} e^{i\omega (n-k)t}dt} \\
            &= \sum_{n=-\infty}^{+\infty}{c_n[n=k]T} \\
            &= Tc_k
            \nonumber
        \end{aligned}
    \end{equation}
\end{large}
我们可以通过平移使得函数$f(t)$成为区间$[-\frac{T}{2},\frac{T}{2}]$上的周期函数,故而有下式成立:
\begin{large}
    \begin{equation}
        \begin{aligned}
            f(t) &= \sum_{k=-\infty}^{+\infty}c_k \cdot e^{i\omega kt} \\
            &= \sum_{k=-\infty}^{+\infty}\left (\frac{1}{T}\int_{-\frac{T}{2}}^{\frac{T}{2}}{f(\xi )e^{-i\omega k\xi}d\xi}\right ) \cdot e^{i\omega kt} \\
            &= \sum_{k=-\infty}^{+\infty}\left (\int_{-\frac{T}{2}}^{\frac{T}{2}}{f(\xi )e^{-i2\pi v\xi}d\xi}\right ) \cdot e^{i2\pi vt} \cdot \frac{1}{T}
            \nonumber
        \end{aligned}
    \end{equation}
\end{large}
其中$v=\frac{k}{T}$,考虑令$T\to \infty$,此时我们可以将一般函数$f$视为具有无穷大周期的周期函数,于是在满足Dirichlet条件的情形下我们有:
\begin{large}
    \begin{equation}
        \begin{aligned}
            f(t) &= \lim_{T \to \infty} \sum_{k=-\infty}^{+\infty}\left (\int_{-\frac{T}{2}}^{\frac{T}{2}}{f(\xi )e^{-i2\pi v\xi}d\xi}\right ) \cdot e^{i2\pi vt} \cdot \frac{1}{T} \\
            &= \int_{-\infty}^{+\infty}\left (\int_{-\infty}^{+\infty}{f(\xi )e^{-i2\pi v\xi}d\xi}\right ) e^{i2\pi vt} dv
            \nonumber
        \end{aligned}
    \end{equation}
\end{large}
令$2\pi v=\omega$就可以得到如下熟悉的傅里叶变换形式:
\begin{large}
    \begin{equation}
        \begin{aligned}
            f(t) =\frac{1}{2\pi}\int_{-\infty}^{+\infty}\left (\int_{-\infty}^{+\infty}{f(\xi )e^{-i\omega\xi}d\xi}\right ) e^{i\omega t} d\omega
            \nonumber
        \end{aligned}
    \end{equation}
\end{large}
于是我们可以由此定义傅里叶变换以及傅里叶逆变换:
\begin{large}
    \begin{equation}
        \begin{aligned}
            &F(\omega)=\int_{-\infty}^{+\infty}f(t)e^{-i\omega t}dt \\
            &f(t)=\frac{1}{2\pi}\int_{-\infty}^{+\infty}F(\omega)e^{i\omega t}d\omega
            \nonumber
        \end{aligned}
    \end{equation}
\end{large}

\section{算法竞赛中的数论与组合数学}

\end{document}